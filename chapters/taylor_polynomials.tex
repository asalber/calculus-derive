% Author: Alfredo Sánchez Alberca (asalber@ceu.es)
\chapter{Taylor polynomials}

% \section{Theoretical foundations}
% A veces, las funciones elementales como las trigonométricas, las exponenciales y las logarítmicas, o composiciones de
% las mismas, son difíciles de tratar y suelen aproximarse mediante polinomios que son funciones mucho más simples y con
% muy buenas propiedades, ya que son continuas y derivables (a cualquier orden) en todos los reales.
% 
% \subsection{Polinomios de Taylor de funciones de una variable}
% \begin{definicion}[Polinomio de Taylor]
% Dada una function $f(x)$, $n$ veces derivable en un punto $a$, se llama \emph{polinomio de Taylor} de orden $n$ para $f$
% en $a$, al polinomio
% \[
% P_{n,f,a}(x)=f(a)+f'(a)(x-a)+\frac{f''(a)}{2!}(x-a)^2+\cdots+\frac{f^{(n}(a)}{n!}(x-a)^n= \sum_{i=0}^{n}\frac{f^{(i}(a)}{i!}(x-a)^i.
% \]
% \end{definicion}
% 
% Este polinomio es el polinomio de grado menor o igual que $n$ que mejor aproxima a $f$ en un entorno del punto $a$, y
% por tanto, si $x$ está próximo a $a$, $f(x)\approx P_{n,f,a}(x)$. 
% Además, cuanto mayor es el grado del polinomio, mejor es la aproximación, tal y como se muestra en el ejemplo de la
% figura~\ref{g:polinomios}.
% \begin{figure}[h!]
% \begin{center}
% \scalebox{1}{\input{img/polinomios_taylor/polinomios_seno}}
% \caption{Polinomios de Taylor de distintos grados para la function $\sin x$ en el punto 0.}
% \label{g:polinomios}
% \end{center}
% \end{figure}
% 
% \subsubsection*{Polinomio de Mc Laurin}
% Cuando nos interesa aproximar una function en un entorno del 0, la ecuación del polinomio de Taylor resulta especialmente simple:
% \[
% P_{n,f,0}(x)=f(0)+f'(0)x+\frac{f''(0)}{2!}x^2+\cdots+\frac{f^{(n}(0)}{n!}x^n= \sum_{i=0}^{n}\frac{f^{(i}(0)}{i!}x^i,
% \]
% y este polinomio se conoce como \emph{polinomio de Mc Laurin} de orden $n$ de $f$.
% 
% \subsubsection*{Resto de Taylor}
% Los polinomios de Taylor nos permiten calcular el valor aproximado de una function en un entorno de un punto, pero normalmente el valor que proporciona el polinomio de Taylor difiere del valor real de la function, es decir, se comete un error en la aproximación. Dicho error se conoce como el \emph{resto de Taylor} de orden $n$ para $f$ en $a$, y es
% \[
% R_{n,f,a}(x)=f(x)-P_{n,f,a}(x).
% \]
% 
% El resto mide el error cometido al aproximar $f(x)$ mediante $P_{n,f,a}(x)$ y nos permite expresar la function $f$ como la suma de un polinomio de Taylor más su resto correspondiente:
% \[
% f(x)=P_{n,f,a}(x)+R_{n,f,a}(x).
% \]
% Esta última expresión se conoce como \emph{fórmula de Taylor} de orden $n$ para $f$ en el punto $a$.
% 
% \subsubsection*{Forma de Lagrange del resto}
% Normalmente, cuando se aproxima una function mediante un polinomio de Taylor, no se conoce el error cometido en la aproximación. No obstante, es posible acotar dicho error de acuerdo al siguiente teorema.
% 
% \begin{teorema}[Resto de Lagrange]
% Sea $f$ una function para la que las $n+1$ primeras derivatives están definidas en el intervalo $[a,x]$. Entonces existe un
% $t\in(a,x)$ tal que el resto de Taylor de orden $n$ para $f$ en el punto $a$ viene dado por
% \[
% R_{n,f,a}(x)=\frac{f^{(n+1}(t)}{(n+1)!}(x-a)^{n+1}.
% \]
% \end{teorema}
% Esta expresión se conoce como \emph{forma de Lagrange del resto}.
% 
% Este teorema nos permite acotar el resto en valor absoluto, ya que una vez fijado el valor de $x$ donde queremos aproximar el valor de la function, el resto en la forma de Lagrange es una function que sólo depende de $t$. Puesto que $t\in (a,x)$, basta con encontrar el máximo del valor absoluto de esta function en dicho intervalo para tener una cota del error cometido.
% 
% \subsection{Polinomios de Taylor de funciones de varias variables}
% Los polinomios de Taylor pueden generalizarse a funciones de más de una variable. Así, por ejemplo, si $f$ es un field
% escalar, el \emph{polinomio de Taylor} de primer grado de $f$ alrededor de un punto $a$ es
% \begin{align*}
% P^2_{f,a}(\mathbf{v})&=f(a)+\nabla f(a)\mathbf{v},
% \end{align*}
% y el de segundo grado es
% \begin{align*}
% P^2_{f,a}(\mathbf{v})&=f(a)+\nabla f(a)\mathbf{v}+\frac{1}{2}\mathbf{v}\nabla^2f(a)\mathbf{v}.
% \end{align*}
% 
% Para el caso particular de funciones de dos variables $f(x,y$ y un punto $a=(x_0,y_0)$, 
% \begin{multline*}
% P^2_{f,a}(x,y) = f(a)+\frac{\partial f(a)}{\partial x}(x-x_0)+\frac{\partial f(a)}{\partial y}(y-y_0)+\\
% +\frac{1}{2}\left(\frac{\partial^2 f(a)}{\partial x^2}(x-x_0)^2 + 2\frac{\partial^2 f(a)}{\partial y\partial x}
% (x-x_0)(y-y_0) + \frac{\partial^2 f(a)}{\partial y^2}(y-y_0)^2\right)
% \end{multline*}
% 
% 
% \newpage

\section{Solved exercises}

\begin{enumerate}[leftmargin=*]
\item Compute the Taylor polynomials of the function $f(x)=\log x$ at the point $x=1$, up to order 4 and plot their graphs.
Which polynomial approximates better the function in a neighbourhood of $1$?

\begin{indication}
\begin{enumerate}
\item Define the function in the Algebra window entering the expression \command{f(x):=log(x)}.
\item Open a new graphic window with the menu \menu{Window > New 2d-plot Window} and select the menu \menu{Window -> Tile Vertically} to see the Algebra and the graphic windows at the same time.
\item Click the button \button{Plot} in the graphic window.
\item Select the name of the function in the Algebra window and select the menu \menu{Calculus > Taylor Series}.
\item In the dialog shown enter the $1$ in the \field{Expansion Point} field, enter 0 in the \field{Order} field and click the button \button{Simplify}.
\item Click the button \button{Plot} in the graphic window.
\end{enumerate}
\end{indication}

\item Compute the value of $\log 1.2$ approximately using the previous polynomials and give the error of the approximation in each case filling the table below.
\[
\begin{tabular}{|c|c|c|c|}
\hline
Point & Grade & Approximation & Error \\
\hline\hline
&  &  &  \\ \hline
&  &  &  \\ \hline
&  &  &  \\ \hline
&  &  &  \\ \hline
&  &  &  \\ \hline
\end{tabular}
\]

\begin{indication}
For each polynomial do the following steps: 
\begin{enumerate}
\item Give a name to the polynomial entering the expression \command{p(x):=\#i}, where \command{\#i} is the label corresponding to the polynomial, or use the the expression \command{p(x):=TAYLOR(f(x),x,1,n)}, where \command{n} is the grade of the polynomial. 
\item Enter the expression \command{p(1.2)} in the Algebra windows and click the button \button{Approximate} to get the approximation. 
\item Enter the expression \command{ABS(p(1.2)-f(1.2))} in the Algebra windows and click the button \button{Approximate} to get the approximation error.
\end{enumerate}
\end{indication}

\item Compute the Maclaurin polynomial of order 3 of the function $\sin(x)$, and use it to approximate the value of $\sin(1/2)$.
Compute the approximation error.

\begin{indication}
\begin{enumerate}
\item Define the function in the Algebra window entering the expression \command{f(x):=sin(x)}.
\item Open a new graphic window with the menu \menu{Window > New 2d-plot Window} and select the menu \menu{Window -> Tile Vertically} to see the Algebra and the graphic windows at the same time.
\item Click the button \button{Plot} in the graphic window.
\item Select the name of the function in the Algebra window and select the menu \menu{Calculus > Taylor Series}.
\item In the dialog shown enter the $1$ in the \field{Expansion Point} field, enter 0 in the \field{Order} field and click the button \button{Simplify}.
\item Click the button \button{Plot} in the graphic window. 
\item Enter the expression \command{p3(x):=TAYLOR(f(x),x,0,3)} in the Algebra window and click the button \button{Simplify}.
\item Enter the expression \command{p3(1/2)} in the Algebra window and click the button \button{Approximate} to get the approximation.
\item Enter the expression \command{ABS(p3(1/2)-f(1.2))} in the Algebra windows and click the button \button{Approximate} to get the approximation error.
\end{enumerate}
\end{indication}

\item Given the function $f(x,y)=\sqrt{xy}$:
\begin{enumerate}
\item Plot its graph.
\begin{indication}
\begin{enumerate}
\item Define the function entering the expression \command{f(x,y):=sqrt(xy)} in the Algebra window.
\item Open a new graphic window with the menu \menu{Window > New 3D-plot Window} and select the menu \menu{Window -> Tile Vertically} to see the Algebra and the graphic windows at the same time.
\item Click the button \button{Plot} in the graphic window.
\end{enumerate}
\end{indication}

\item Compute first degree Taylor polynomial of $f$ at the point $(8,2)$ and plot its graph.
Check that you get the tangent plane to the graph of $f$ at the point $(8,2)$.
\begin{indication}
\begin{enumerate}
\item Enter the expression \command{[8,2,f(8,2)]} in the Algebra window. 
\item Click the button \button{Plot} in the graphic window.
\item Enter the expression \command{p1(x,y):=f(8,2)+f'(8,2)[x-8,y-2]} in the Algebra window and click the button \button{Simplify}.
\item Click the button \button{Plot} in the graphic window.
\end{enumerate}
\end{indication}

\item Use the previous polynomial to approximate the value of $\sqrt{8.02\cdot 1.99}$.
\begin{indication}
Enter the expression \command{p1(8.02,1.99)} in the Algebra window and click the button \button{Approximate}.
\end{indication}

\item Compute the error of the previous approximation. 
\begin{indication}
Enter the expression \command{abs(p1(8.02,1.99)-f(8.02,1.99))} in the Algebra window and click the button \button{Approximate}.
\end{indication}

\item Compute second degree Taylor polynomial of $f$ at the point $(8,2)$ and plot its graph.
\begin{indication}
\begin{enumerate}
\item Enter the expression \command{p2(x,y):=p1(x,y)+1/2[x-8,y-2]f''(8,2)[x-8,y-2]} in the Algebra window and click the button \button{Simplify}.
\item Click the button \button{Plot} in the graphic window.
\end{enumerate}
\end{indication}

\item Use the previous polynomial to approximate the value of $\sqrt{8.02\cdot 1.99}$.
\begin{indication}
Enter the expression \command{p2(8.02,1.99)} in the Algebra window and click the button \button{Approximate}.
\end{indication}

\item Compute the error of the previous approximation. 
Check that the error of the second degree Taylor polynomial is less than the error with the first degree Taylor polynomial. 
\begin{indication}
Enter the expression \command{abs(p2(8.02,1.99)-f(8.02,1.99))} in the Algebra window and click the button \button{Approximate}.
\end{indication}
\end{enumerate}
\end{enumerate}


\section{Proposed exercises}

\begin{enumerate}[leftmargin=*]
\item Given the function $f(x)=\sqrt{x+1}$:
\begin{enumerate}
\item Compute the 4th degree Taylor polynomial of $f$ at point $x=0$.
\item Approximate the value of $\sqrt{1.02}$ using the 2nd degree and 4th degree Taylor polynomials at $x=0$ and compute the approximation error en each case.
\end{enumerate}

\item Given the functions $f(x)=e^x$ and $g(x)=\cos x$:
\begin{enumerate}
\item Compute the 2nd degree Maclaurin polynomials of $f$ and $g$.
\item Use the previous polynomials to compute 
\[ \lim_{x\rightarrow 0}\frac{e^x-\cos x}{x}.\]
\end{enumerate}

\item Compute the value of $\log(0.09^3+0.99^3)$ approximately using:
\begin{enumerate}
\item A first degree Taylor polynomial.
\item A second degree Taylor polynomial.
\end{enumerate}
\end{enumerate}
