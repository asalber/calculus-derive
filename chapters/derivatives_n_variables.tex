% Author: Alfredo Sánchez Alberca (asalber@ceu.es)
\chapter{Several variables differentiable calculus}

% \section{Theoretical foundations}
% El concepto de derivative es uno de los más importantes del Cálculo
% pues resulta de gran utilidad en el estudio de funciones y tiene
% multitud de aplicaciones. En esta práctica introducimos este
% concepto de derivative parcial en funciones de varias variables y presentamos algunas de sus aplicaciones.
% 
% \subsection{Derivadas parciales de una function de $n$ variables}
% Recordemos que la derivative de una function de una única variable en
% un punto $x=a$ se define como la tasa de variación instantánea de la
% function en dicho punto. Si llamamos $h$ al incremento en la
% variable, la derivative de la function en $x=a$ se nota como $f'(a)$ ó
% $\dfrac{{df}} {{dx}}(a)$, y se calcula como:
% 
% \[
% f'(a) = \frac{df}{dx}(a) = \mathop {\lim
% }\limits_{h \to 0} \frac{{f(a + h) - f(a)}} {h}
% \]
% 
% Y, en general, para cualquier $x$ en un conjunto en el que la
% function de una única variable sea derivable, puede definirse la
% function derivative $f'(x)$ ó $\dfrac{df}{dx}$,
% mediante el límite:
% 
% \[
% f'(x) = \frac{df}{dx} = \mathop {\lim }\limits_{h
% \to 0} \frac{{f(x + h) - f(x)}} {h}
% \]
% que, no obstante, se calcula aplicando las adecuadas reglas de
% derivación, más bien que acudiendo a la resolución directa del
% límite.
% 
% De igual forma, si tenemos una function de $n$ variables
% $f(\vec{x})=f(x_1,x_2,...,x_n)$, se pueden definir esas mismas tasas
% de variación instantáneas con respecto a cada una de las variables
% en un punto $\vec{a}=(a_1,a_2,...,a_n)$, para obtener las
% \emph{Derivadas Parciales} de la function con respecto a cada una de
% las variables en el punto $\vec{x}=\vec{a}$, que se notarán como
% $f_{x_i}(\vec{a})$, o bien por $\dfrac{\partial f}{\partial
% x_i}(\vec{a})$:
% 
% \[
% f_{x_i}(\vec{a})=\dfrac{\partial f}{\partial
% x_i}(\vec{a})=\lim_{h\rightarrow
% 0}\frac{f(a_1,\ldots,a_i+h,\ldots,a_n)-f(a_1,\ldots,a_i,\ldots,a_n)}{h},
% \]
% 
% Y la \emph{Función Derivada Parcial} con respecto a cualquiera de
% las variables $x_i$, para todos los $\vec{x}$ de un conjunto, que se
% notará como $f_{x_i}$, o $\dfrac{\partial f}{\partial x_i}$:
% 
% \[
% f_{x_i}=\dfrac{\partial f}{\partial x_i}=\lim_{h\rightarrow
% 0}\frac{f(x_1,\ldots,x_i+h,\ldots,x_n)-f(x_1,\ldots,x_i,\ldots,x_n)}{h},
% \]
% 
% Lo cual, a efectos de cálculo, se resume en que la derivative parcial
% de una function de varias variables se obtiene como la derivative de
% una function de una única variable, que es aquella con respecto a la
% que se deriva, y constantes el resto. Como consecuencia, las reglas
% de derivación también se aplican en el cálculo de las derivatives
% parciales de este tipo de funciones, sin más que considerar
% constantes todas las variables con respecto a las que no estamos
% derivando.
% 
% Si para las funciones de una variable la derivative en un punto $a$
% tiene una interpretación gráfica sencilla como la pendiente de la
% recta tangente a la gráfica de la function en el punto $(a,f(a))$,
% también para funciones de dos variables la derivative parcial con
% respecto a $x$ en el punto $\vec{a}=(x_0,y_0)$:
% 
% \[
% \frac{{\partial f}} {{\partial x}}(x_0 ,y_0 ) = \mathop {\lim
% }\limits_{h \to 0} \frac{{f(x_0  + h,y_0 ) - f(x_0 ,y_0 )}} {h}
% \]
% representa la pendiente de la recta tangente a la curva que se
% obtiene al cortar la gráfica de la function en el punto
% $(x_0,y_0,z_0=f(x_0,y_0))$ mediante el plano en el que $y$ permanece
% constante e igual a $y_0$, y tan sólo varía el valor de $x$. E
% igualmente, la derivative parcial con respecto a $y$ será la pendiente
% de la recta tangente a la curva que se obtiene al cortar la gráfica
% de la function en $(x_0,y_0,z_0=f(x_0,y_0))$ mediante el plano
% $x=x_0$.
% 
% \begin{figure}[h!]
% \begin{center}
% \includegraphics[scale=0.4]{img/derivatives_varias_variables/tangentesuperficie}
% \caption{La derivative parcial de una function $f(x,y)$ con respecto a
% $x$ en el punto $(x_0,y_0)$, como la pendiente de la recta tangente
% a la curva descrita por la intersección de la superficie de $f$ y el
% plano de ecuación $y=y_0$.}
% \end{center}
% \end{figure}
% 
% 
% \subsection{Derivadas parciales sucesivas de una function de $n$
% variables}
% 
% De la misma forma que en las funciones de una variable, mediante los
% límites que las definen, siempre y cuando existan, obtenemos las
% segundas, terceras y derivatives de cualquier orden. Es decir, si $f$
% es una function real de $n$ variables, con sus correspondientes
% derivatives parciales, a su vez también las mismas son funciones de
% $n$ variables que, en determinadas condiciones, podrán derivarse de
% nuevo con respecto a sus $n$ variables para obtener derivatives
% parciales segundas, y así sucesivamente hasta órdenes superiores de
% derivación.
% 
% Para la derivative parcial de segundo orden se utiliza la notación
% $f_{x_ix_j}$ ó $\dfrac{{\partial ^2 f}} {{\partial x_j \partial x_i
% }}$:
% 
% 
% \[
% f_{x_i x_j }  = \frac{{\partial ^2 f}} {{\partial x_j \partial x_i
% }} = \frac{\partial } {{\partial x_j }}\left( {\frac{{\partial f}}
% {{\partial x_i }}} \right)
% \]
% 
% Por ejemplo, para una function de dos variables $f(x,y)$, tenemos dos derivatives parciales de primer orden, que siguen
% siendo funciones de las variables $x$ e $y$:
% \[
% \frac{\partial f}{\partial x}(x,y)\qquad \frac{\partial f}{\partial y}(x,y) 
% \] y cuatro diferentes de segundo orden, que también serán funciones de $x$ e $y$, aunque ya no se
% refleje para aligerar la notación:
% \[
% \frac{\partial}{\partial x}\left({\frac{\partial f}{\partial x}}\right) = \frac{\partial^2 f}{\partial x^2}
% \]
% \[
% \frac{\partial}{\partial y}\left({\frac{\partial f}{\partial x}}\right) = \frac{\partial^2 f}{\partial y\partial x}
% \]
% \[
% \frac{\partial}{\partial x}\left({\frac{\partial f}{\partial y}}\right) = \frac{\partial^2 f}{\partial x\partial y}
% \]
% \[
% \frac{\partial}{\partial y}\left({\frac{\partial f}{\partial y}}\right) = \frac{\partial^2 f}{\partial y^2}
% \]
% 
% La primera y la última reciben el nombre de derivatives segundas, mientras que la segunda y tercera se denominan derivatives
% cruzadas.
% 
% Si procedemos con las derivatives parciales de tercer orden tendríamos ocho diferentes, y el número es más amplio con
% funciones de tres o más variables.
% No obstante, el teorema conocido con \emph{Teorema de Schwartz de las Derivadas Cruzadas} reduce el número de derivatives
% parciales diferentes:
% 
% \begin{teorema}[Schwartz]
% Si $f$ es una function de $n$ variables con derivatives parciales segundas cruzadas continuas en un conjunto abierto que
% contiene al punto $a$, entonces en dicho punto se cumple
% \[
% \frac{\partial ^2 f}{\partial x_i \partial x_j } = \frac{\partial ^2 f}{\partial x_j \partial x_i }
% \] 
% e igual con las derivatives cruzadas de tercer y superior orden.
% \end{teorema}
% 
% Es decir, si se cumplen las hipótesis del teorema de Schwartz concluimos que, a efectos de cálculo, tan sólo importa el
% número de veces que se deriva respecto a cada variable, pero no el orden de la derivación.
% 
% \subsection{Vector gradiente y matriz hessiana}
% A partir de las derivatives parciales de primer orden de un field escalar, podemos definir the following vector:
% 
% \begin{definicion}[Vector gradiente]
% Dado un field escalar $f(x_1,\ldots,x_n)$, se llama \emph{gradiente} de $f$, y se escribe $\nabla f$, a la function que a
% cada punto $a=(a_1,\ldots,a_n)$ le asigna el vector cuyas coordenadas cartesianas son las derivatives parciales de $f$ en
% $a$,
% \[
% \nabla f(a)=\left(\frac{\partial f}{\partial x_1}(a),\ldots,\frac{\partial f}{\partial x_n}(a)\right).
% \]
% \end{definicion}
% 
% El vector gradiente en un punto dado tiene la misma magnitud y dirección que la velocidad máxima de variación de la
% function en ese punto, por lo que $\nabla f(a)$ indica la dirección de máximo crecimiento de $f$ en el punto $a$,
% mientras que $-\nabla f(a)$ indica la dirección de máximo decrecimiento.
% 
% Y a partir de las derivatives parciales de segundo orden, podemos definir the following matriz:
% 
% \begin{definicion}[Matriz hessiana]
% Dada una function de varias variables $f(x_1,\ldots,x_n)$, para la que existen todas sus derivatives parciales de segundo
% orden en un punto $a=(a_1,\ldots,a_n)$, se define la \emph{matriz hessiana} de $f$ en $a$, y se nota $\nabla^2f(a)$, como la
% matriz cuadrada cuyos elementos son
% \[
% \nabla^2f(a)=\left(
% \begin{array}{cccc}
% \dfrac{\partial^2 f}{\partial x_1^2}(a) & 
% \dfrac{\partial^2 f}{\partial x_1 \partial x_2}(a) &
% \cdots &
% \dfrac{\partial^2 f}{\partial x_1 \partial x_n}(a)\\
% \dfrac{\partial^2 f}{\partial x_2 \partial x_1}(a) &
% \dfrac{\partial^2 f}{\partial x_2^2}(a) & 
% \cdots &
% \dfrac{\partial^2 f}{\partial x_2 \partial x_n}(a)\\
% \vdots & \vdots & \ddots & \vdots \\
% \dfrac{\partial^2 f}{\partial x_n \partial x_1}(a) &
% \dfrac{\partial^2 f}{\partial x_n \partial x_2}(a) &
% \cdots &
% \dfrac{\partial^2 f}{\partial x_n^2}(a)
% \end{array}
% \right).
% \]
% Al determinante de esta matriz se le llama \emph{hessiano} de $f$ en $a$, y se nota $Hf(a)=|\nabla^2f(a)|$.
% \end{definicion}
% 
% Entre las utilidades de la matriz Hessiana y el hessiano está la determinación de los extremos relativos y los puntos de
% silla de una function. 
% 
% \subsection{Derivada direccional}
% Ya se ha visto que la derivatives parciales miden la tasa de variación instantánea de la function con respecto a cada uno
% de sus variables, es decir, en la dirección de cada uno de los ejes de coordenadas. 
% Pero, \emph {¿qué pasa si nos movemos en cualquier otra dirección?}
% La tasa de variación instantánea de $f$ en un punto $a$ en la dirección de un vector unitario cualquiera $u$ se conoce
% como \emph{derivative direccional}.
% 
% \begin{definicion}[Derivada direccional]
% Dado un field escalar $f$ de $\mathbb{R}^n$, un punto $a$ y un vector unitario $\mathbf{u}$ en ese espacio, el límite
% \[
% f'_{\mathbf{u}}(a) = \lim_{h\rightarrow 0}\frac{f(a+h\mathbf{u})-f(a)}{h},
% \] 
% cuando existe, se llama \emph{derivative direccional} de $f$ en el punto $a$ en la dirección de $\mathbf{u}$.
% \end{definicion}
% Se cumple 
% \[
% f'_{\mathbf{u}}(a) =\nabla f(a)\cdot \mathbf{u}.
% \]
% Obsérvese que las derivatives parciales son las derivatives direccionales en las direcciones de los vectores coordenados.
% 
% \subsection{Derivación implícita}
% Si se sabe que la ecuación 
% \[
% f(x,y)=0
% \]
% define a $y$ como function de $x$, $y=h(x)$, alrededor de cierto valor $x=x_0$ para el que $y=h(x_0)=y_0$, entonces, si se toma la trayectoria $g(x)=(x,h(x))$, la ecuación anterior se puede expresar como
% \[
% (f\circ g)(x) = f(g(x)) = f(x,h(x))=0,
% \]
% de modo que usando la regla de la cadena sobre se tiene
% \[
% (f\circ g)'(x) = \nabla f(g(x))\cdot g'(x) = \left(\frac{\partial f}{\partial x}, \frac{\partial f}{\partial y}\right)\cdot (1,h'(x)) = 
% \frac{\partial f}{\partial x}+\frac{\partial f}{\partial y}h'(x) = 0,
% \] 
% de donde se deduce
% \[
% y'=h'(x)=\frac{-\dfrac{\partial f}{\partial x}}{\dfrac{\partial f}{\partial y}}
% \]
% 
% \subsection{Cálculo de extremos}
% Si $a$ es un extremo de un field escalar $f$ de $\mathbb{R}^n$, entonces se cumple que $a$ es un punto crítico de $f$,
% es decir,
% \[
% \nabla f(P) = 0,
% \]
% Sin embargo, no todos los puntos críticos son extremos relativos. 
% Existen puntos como el de la figura~\ref{g:punto_silla} donde se anula el gradiente y que no son puntos de máximo, ni de
% mínimo.
% Estos puntos se conocen como \emph{puntos de silla}. 
% \begin{figure}[htp]
% \begin{center}
% \scalebox{1}{\input{img/derivatives_varias_variables/punto_silla}}.
% \caption{Gráfica de la function $f(x,y)=x^2-y^2$, que tiene un punto de silla en $(0,0)$. }
% \label{g:punto_silla}
% \end{center}
% \end{figure}
% 
% Afortunadamente, es posible determinar si un punto crítico es un extremo relativo o un punto de silla a partir de la
% matriz Hessiana y el hessiano. 
% 
% \begin{teorema}
% Dado un punto crítico $a$ de un field escalar $f$ que tiene matríz hessiana $Hf(a)$, se cumple
% \begin{itemize}
% \item Si $\nabla^2f(a)$ es definido positivo entonces $f$ tiene un mínimo relativo en $a$.
% \item Si $\nabla^2f(a)$ es definido negativo entonces $f$ tiene un máximo relativo en $a$.
% \item Si $\nabla^2f(a)$ es indefinido entonces $f$ tiene un punto de silla en $a$.
% \end{itemize}
% \end{teorema}
% En el caso de que $\nabla^2f(a)$ sea semidefinido, no se puede obtener ninguna conclusión y hay que recurrir a derivatives
% parciales de orden superior.
% 
% En el caso particular de un field escalar de dos variables se tiene
% \begin{teorema}
% Dado un punto crítico $a=(x_0,y_0)$ de un field escalar $f(x,y)$ que tiene matríz hessiana $\nabla^2f(a)$, se cumple
% \begin{itemize}
% \item Si $Hf(a)>0$ y $\dfrac{\partial^2 f}{\partial x^2}(a)>0$ entonces $f$ tiene un mínimo relativo en $a$.
% \item Si $Hf(a)>0$ y $\dfrac{\partial^2 f}{\partial x^2}(a)<0$ entonces $f$ tiene un máximo relativo en $a$.
% \item Si $Hf(a)<0$ entonces $f$ tiene un punto de silla en $a$.
% \end{itemize}
% \end{teorema}
% 
% \newpage

\section{Solved exercises}
\begin{enumerate}[leftmargin=*]
\item Compute the tangent line and the normal plane to the trajectory of
\[
f(t)=
\begin{cases}
x=\sin(t),\\
y=\cos(t),\\
z=\sqrt(t),
\end{cases}
\quad t\in \mathbb{R};
\] 

at the time $t=1$ and plot them.
  
\begin{indication}
To plot the trajectory:
\begin{enumerate}
\item Define the function entering the expression \command{f(t):=[sin(t),cos(t),sqrt(t)]} in the Algebra window.
\item Open a new graphic window with the menu \menu{Window > New 3D-plot Window} and select the menu \menu{Window -> Tile Vertically} to see the Algebra and the graphic windows at the same time.
\item Click the button \button{Plot} in the graphic window.
\end{enumerate}
To compute the tangent line and plot its graph:
\begin{enumerate}
\item Enter the expression \command{f(1)} in the Algebra window.
\item Click the button \button{Plot} in the graphic window.
\item Enter the expression \command{f(1)+tf'(1)} in the Algebra window and click the button \button{Simplify}.
\item Click the button \button{Plot} in the graphic window.
\end{enumerate}
To compute the normal plane and plot its graph:
\begin{enumerate}
\item Enter the expression \command{([x,y,z]-f(1))f'(1)=0} in the Algebra window and click the button \button{Simplify}.
\item Click the button \button{Plot} in the graphic window.
\end{enumerate}
\end{indication}  
  
\item Given the function $f(x,y)=y^2-x^2$,
\begin{enumerate}
\item Plot its graph.
\begin{indication}
\begin{enumerate}
\item Enter the expression \command{f(x,y):=y\^{}2-x\^{}2} in the Algebra window.
\item Open a new graphic window with the menu \menu{Window > New 3D-plot Window} and select the menu \menu{Window -> Tile Vertically} to see the Algebra and the graphic windows at the same time.
\item Click the button \button{Plot} in the graphic window.
\end{enumerate}
\end{indication}

% \item Plot the level curves $f(x,y)=k$, for $k=1,\ldots,5$.
% \begin{indication}
% For each value of $k$:
% \begin{enumerate}
% \item Enter the expression \command{f(x,y)=k}.
% \item Open a new graphic window with the menu \menu{Window > New 2D-plot Window} and select the menu \menu{Window -> Tile Vertically} to see the Algebra and the graphic windows at the same time.
% \item Click the button \button{Plot} in the graphic window.
% \end{enumerate}
% \end{indication}

\item Plot the plane with equation $x=1$. 
What shape has the curve that results from the intersection of this plane and the graph of $f$?
\begin{indication}
\begin{enumerate}
\item Enter the expression \command{x=1} in the Algebra window.
\item Click the button \button{Plot} in the 3D graphic window.
\end{enumerate}
\end{indication}

\item Compute the derivative of $f(1,y)$ at $y=2$.
\begin{indication}
\begin{enumerate}
\item Enter the expression \command{f(1,y)} in the Algebra window.
\item Select the menu \menu{Calculus > Differentiate} or click the button \button{Find Derivative}.
\item In the dialog shown click the button \button{Simplify}.
\item Select the menu \menu{Simplify > Variable Substitution} or click the button \button{Variable Substitution}.
\item In the dialog shown enter the value 2 in the field \field{New Value} and click the button \button{Simplify}.
\end{enumerate}
\end{indication}

\item Plot the plane with equation $y=2$. 
What shape has the curve that results from the intersection of this plane and the graph of $f$?
\begin{indication}
\begin{enumerate}
\item Enter the expression \command{y=2} in the Algebra window.
\item Click the button \button{Plot} in the 3D graphic window.
\end{enumerate}
\end{indication}

\item Compute la derivative of $f(x,2)$ at $x=1$.
\begin{indication}
\begin{enumerate}
\item Enter the expression \command{f(x,2)} in the Algebra window.
\item Select the menu \menu{Calculus > Differentiate} or click the button \button{Find Derivative}.
\item In the dialog shown click the button \button{Simplify}.
\item Select the menu \menu{Simplify > Variable Substitution} or click the button \button{Variable Substitution}.
\item In the dialog shown enter the value 1 in the field \field{New Value} and click the button \button{Simplify}.
\end{enumerate}
\end{indication}

\item Compute the partial derivatives of $f$ at the point $(1,2)$. 
What conclusions can you draw?
\begin{indication}
For the partial derivative of $f$ with respect to $x$:
\begin{enumerate}
\item Enter the expression \command{f(x,y)} or select the expression corresponding to $f$ in the Algebra window.
\item Select the menu \menu{Calculus > Differentiate} or click the button \button{Find Derivative}.
\item In the dialog shown select the variable \option{x} in the drop-down list and click the button \button{Simplify}.
\item Select the menu \menu{Simplify > Variable Substitution} or click the button \button{Variable Substitution}.
\item In the dialog shown select the variable \option{x} in the list \field{Variables} and enter the value 1 in the field \field{New Value}, then select the variable \option{y} in the list \field{Variables} and enter the value 2 in the field \field{New Value}, and click the button \button{Simplify}.
\end{enumerate}
A faster way of computing the partial derivative is entering the command \command{DIF(f(x,y),x)}

For the partial derivative of $f$ with respect to $y$ repeat the previous steps but selecting the variable \option{y} in the drop-down list of the \menu{Calculus Differentiate} dialog.
\end{indication}
\end{enumerate}


\item  Compute the following partial derivatives:
\begin{enumerate}
\item  $\dfrac{\partial }{\partial V}\dfrac{nRT}{V}.$

\begin{indication}
Enter the expression \command{DIF(nRT/V, V)} in the Algebra window and click the button \button{Simplify}.
\end{indication}

\item  $\dfrac{\partial ^{2}}{\partial x\partial y}e^{x+y}\sin(x/y).$
\begin{indication}
\begin{enumerate}
Enter the expression \command{DIF(DIF(exp(x+y)sin(x/y), y), x)} in the Algebra window and click the button \button{Simplify}.
\end{enumerate}
\end{indication}
\end{enumerate}

\item Given the function $f(x,y)=20-4x^2-y^2$, compute at the point $(2,-3)$:
\begin{enumerate}
\item Gradient.
\begin{indication}
\begin{enumerate}
\item Define the function entering the expression \command{f(x,y):=20-4x\^{}2-y\^{}2} in the Algebra window.
\item Enter the expression \command{f'(2,-3)} in the Algebra window and click the button \button{Simplify}.
\end{enumerate}
\end{indication}

\item Hessian matrix.
\begin{indication}
Enter the expression \command{f''(2,-3)} in the Algebra window and click the button \button{Simplify}.
\end{indication}

\item Hessian.
\begin{indication}
Enter the expression \command{DET(f''(2,-3))} in the Algebra window and click the button \button{Simplify}.
\end{indication}
\end{enumerate}
% 
% 
% \item Dada la function
% \[
% f(x,y,z)=\sin((x^2-y^2)z)
% \]
% \begin{enumerate}
% \item Definir la function.
% \begin{indication}
% Enter the expression \command{f(x,y,z):=sin((x\^{}2-y\^{}2)z)}.
% \end{indication}
% 
% \item Compute su vector gradiente en el punto $(0,-1,\pi/2)$.
% \begin{indication}
% \begin{enumerate}
% \item Enter the expression \command{f'(0,-1,pi/2)}.
% \item Hacer clic sobre el button \button{Simplify}.
% \end{enumerate}
% \end{indication}
% 
% \item Compute su matriz Hessiana en el punto $(0,-1,\pi/2)$.
% \begin{indication}
% \begin{enumerate}
% \item Enter the expression \command{f''(0,-1,pi/2)}.
% \item Hacer clic sobre el button \button{Simplify}.
% \end{enumerate}
% \end{indication}
% 
% \item Comprobar que se cumple el teorema Schwartz de las derivatives para las derivas cruzadas:
% \begin{enumerate}
% \item $\dfrac{{\partial ^3 f}} {{\partial x\partial z\partial y}}$
% \begin{indication}
% \begin{enumerate}
% \item Enter the expression \command{f(x,y,z)}.
% \item Hacer clic sobre el button \button{Hallar una derivative}.
% \item En el cuadro de diálogo que aparece seleccionar la variable $y$ y click the button \button{Simplify}.
% \item Repetir el proceso con la expresión resultante, seleccionando esta vez la variables $z$.
% \item Repetir una vez más el proceso con la expresión resultante, seleccionando esta vez la variables $x$.
% \end{enumerate}
% \end{indication}
% 
% \item $\dfrac{{\partial ^3 f}} {{\partial z\partial x\partial y}}$
% \begin{indication}
% \begin{enumerate}
% \item Enter the expression \command{f(x,y,z)}.
% \item Hacer clic sobre el button \button{Hallar una derivative}.
% \item En el cuadro de diálogo que aparece seleccionar la variable $y$ y click the button \button{Simplify}.
% \item Repetir el proceso con la expresión resultante, seleccionando esta vez la variables $x$.
% \item Repetir una vez más el proceso con la expresión resultante, seleccionando esta vez la variables $z$.
% \end{enumerate}
% \end{indication}
% \end{enumerate}
% 
% ¿Puedes predecir el valor de $\dfrac{{\partial ^3 f}} {{\partial y\partial x\partial z}}$?
% \end{enumerate}
% 
\item Compute the normal line and the tangent plane to the surface $S: x+2y-\log z +4 =0$ at the point $(0,-2,1)$ and plot them.
\begin{indication}
To plot the surface:
\begin{enumerate}
\item Define the function entering the expression \command{f(x,y,z):=x+2y-log(z)+4} in the Algebra window.
\item Enter the expression \command{f(x,y,z)=0} in the Algebra window.
\item Select the menu \menu{Solve > Expression}.
\item In the dialog shown select the variable \option{z} in the list \field{Solution Variables}, check the option \option{Real} in the field \field{Solution Domain} and click the button \button{Solve}.
\item Open a new graphic window with the menu \menu{Window > New 3D-plot Window} and select the menu \menu{Window -> Tile Vertically} to see the Algebra and the graphic windows at the same time.
\item Click the button \button{Plot} in the graphic window.
\end{enumerate}
To compute normal line and plot its graph:
\begin{enumerate}[resume]
\item Enter the expression \command{[0,-2,1]+tf'(0,-2,1)} and click the button \button{Simplify}.
\item Click the button \button{Plot} in the graphic window.
\end{enumerate}
To compute the tangent plane and plot its graph:
\begin{enumerate}[resume]
\item Enter the expression \command{([x,y,z]-[0,-2,1])f'(0,-2,1)=0} in the Algebra window and click the button \button{Simplify}.
\item Click the button \button{Plot} in the graphic window.
\end{enumerate}
\end{indication}
% 
% \item La ecuación $x+y-2e^y+2=0$ define implícitamente una function $y=f(x)$ alrededor del punto $(0,0)$. Compute
% $f'(0)$.
% \begin{indication}
% \begin{enumerate}
% \item Enter the expression \command{x+y-2\#e\^{}y+2}.
% \item Hacer clic sobre el button \button{Hallar una derivative}.
% \item En el cuadro de diálogo que aparece seleccionar la variable $x$ y click the button \button{Simplify}. 
% \item Seleccionar de nuevo la expresión inicial.
% \item Hacer clic sobre el button \button{Hallar una derivative}.
% \item En el cuadro de diálogo que aparece seleccionar la variable $y$ y click the button \button{Simplify}.
% \item Enter the expression \command{-\#i/\#j} donde \command{\#i} es la etiqueta de la expresión de la derivative
% parcial con respecto a $x$ y \command{\#j} es la etiqueta de la expresión de la derivative parcial con respecto a $y$.
% \item Hacer clic sobre el button \button{Sustituir}.
% \item En el cuadro de diálogo que aparece, seleccionar $x$ e introducir el valor $0$, seleccionar $y$ e
% introducir el valor $0$, y click the button \button{Simplify}.
% \end{enumerate}
% \end{indication}
% 
\item Compute directional derivative of the function $h(x,y)= 3x^2+y$ at the point $(0,0)$, along the vector $(1,1)$.
\begin{indication}
\begin{enumerate}
\item Define the function entering the expression \command{h(x,y):=3x\^{}2+y} in the Algebra window.
\item Enter the expression \command{h'(0,0)SIGN([1,1])} in the Algebra window and click the button \button{Simplify}.
\end{enumerate}
\end{indication}

\item Given the function $f(x,y)=x^3+y^3-3xy$:
\begin{enumerate}
\item Plot its graph. 
Looking at the graph, can you determine the relative extrema of $f$?
\begin{indication}
\begin{enumerate}
\item Define the function entering the expression \command{f(x,y):=x\^{}3+y\^{}3-3xy}.
\item Open a new graphic window with the menu \menu{Window > New 3D-plot Window} and select the menu \menu{Window -> Tile Vertically} to see the Algebra and the graphic windows at the same time.
\item Click the button \button{Plot} in the graphic window.
\end{enumerate}
\end{indication}

\item Compute critical points of $f$.
\begin{indication}
The critical points are the points that make zero the gradient. 
\begin{enumerate}
\item Enter the expression \command{f'(x,y)=0} in the Algebra window.
\item Select the menu \menu{Solve > Expression}.
\item In the dialog shown select the variables \option{x} and \option{y} in the list \field{Solution Variables}, check the option \option{Real} in the field \field{Solution Domain} and click the button \button{Solve}.
\end{enumerate}
\end{indication}

\item Determine the relative extrema and the saddle points of $f$.
\begin{indication}
For each critical point $(a,b)$ you have to compute the Hessian matrix and the Hessian:
\begin{enumerate}
\item Enter the expression \command{f''(a,b)} in the Algebra window and click the button \button{Simplify}.
\item Enter the expression \command{DET(f''(a,b))} in the Algebra window and click the  button \button{Simplify}.
\end{enumerate}
If the Hessian is positive and the second partial derivative of $f$ with respect to $x$ two times is positive the function has a relative minimum at the critical point.
If the Hessian is positive and the second partial derivative of $f$ with respect to $x$ two times is negative the function has a relative maximum at the critical point.
If the Hessian is negative the function has a saddle point at the critical point.
\end{indication}
\end{enumerate}

\end{enumerate}


\section{Proposed exercises}
\begin{enumerate}[leftmargin=*]
\item A spaceship, traveling near the sun, is in trouble.
The temperature at position $(x,y,z)$ is given by 
\[T(x,y,z)=\mbox{e}^{-x^2-2y^2-3z^2},\]
where the variables are measured
in thousands of kilometers, and we assume the sun is at position $(0,0,0)$.
If the ship is at position $(1,1,1)$, find the direction in which it should move so that the temperature will decrease as fast as possible.

\item Compute the gradient, the Hessian matrix and the Hessian of the function
\[
g(x,y,z) = \frac{x}{\sqrt{x^2+y^2+z^2}^3}
\]
at the point $(1,1,1)$ and at the point $(0,3,4)$.

\item Determine the points of the ellipsoid $S: x^2+2y^2+z^2=1$ where the tangent plane is parallel to the plane $\Pi:
x-y+2z^2=0$.

\item Determine the relative extrema of the function 
\[
f(x)=-\frac{y}{9+x^2+y^2}.
\]

\item Compute the directional derivative of the scalar field $f(x,y,z)=x^2-y^2+xyz^3-zx$ at the point $(1,2,3)$ along the vector $(1,-1,0)$.
\end{enumerate}
