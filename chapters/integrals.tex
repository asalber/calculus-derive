% Author: Alfredo Sánchez Alberca (asalber@ceu.es)
\chapter{Integrals}

% \section{Fundamentos teóricos}
% Junto al concepto de derivative, el de integral es otro de los más importantes
% del cálculo matemático. Aunque dicho concepto surgesin principio, como técnica
% para el cálculo de áreas, el teorema fundamental del cálculo establece su
% relación con la derivative, de manera que,sin cierto sentido, la diferenciación
% y la integración son operaciones inversas.
% 
% En esta práctica se introduce el concepto de integral como antiderivative, y
% también el de integral de Riemann, que permite calcular áreas por debajo de
% funciones acotadassin un intervalo.
% 
% \subsection{Primitivas e Integrales}
% \subsubsection*{Función Primitiva}
% 
% Se dice que la función $F(X)$ es una \emph{función antiderivative} de
% $f(x)$ si se verifica que $F'(x)=f(x)$ $\forall x \in \dom f$.
% 
% Como dos funciones que difieransin una constante tienen la misma
% derivative, si $F(x)$ es una función antiderivative de $f(x)$ también lo será toda función de la forma $F(x)+k$ $\forall k \in \mathbb{R}$.\\
% 
% 
% \subsubsection*{Función integral indefinida}
% 
% Se llama \emph{función integral indefinida} de la función $f(x)$ al
% conjunto de todas sus funciones antiderivatives y se representa como:
% 
% \[
% \ \int{f(x)}\,dx=F(x)+C
% \]
% siendo $F(x)$ una función antiderivative de $f(x)$ y $C$ una constante arbitraria.\\
% 
% 
% \subsubsection*{Linealidad de la integral}
% 
% Dadas dos funciones $f(x)$ y $g(x)$ que admiten antiderivative, y una
% constante $k \in \mathbb{R}$ se verifica que:
% 
% \[
% \ \int{(f(x)+g(x))}\,dx=\int{f(x)}\,dx+\int{g(x)}\,dx
% \]
% y:
% \[
% \ \int{kf(x)}\,dx=k\int{f(x)}\,dx
% \]\\
% 
% 
% \subsection{Integral de Riemann}
% 
% Se llama \emph{partición} de un intervalo $[a,b]\subset\mathbb{R}$,
% a una colección finita de puntos del intervalo,
% $P=\{x_{0},x_{1},...,x_{n}\}$,  tales que
% $x_{0}=a<x_{1}<...<x_{n}=b$, con lo que el intervalo $[a,b]$ queda
% divididosin $n$ subintervalos $[x_{i},x_{i+1}]$, $i=0,...,n-1$.
% 
% Dada una función $f:[a,b]\rightarrow\mathbb{R}$ acotada y una
% partición $P=\{x_{0},x_{1},...,x_{n}\}$ de $[a,b]$, se llama
% \emph{suma inferior} de $f$sin relación a $P$, y se designa por
% $L(P,f)$, a:
% 
% \[
% \ L(P,f)=\sum_{i=1}^{n} m_{i}(x_{i}-x_{i-1})
% \]
% donde $  m_{i}=\inf\{f(x):x_{i-1}\leq x \leq x_{i}\}$.
% 
% Análogamente se llama \emph{suma superior} de $f$sin relación a $P$,
% y se designa por $U(P,f)$, a:
% 
% \[
% \ U(P,f)=\sum_{i=1}^{n} M_{i}(x_{i}-x_{i-1})
% \]
% donde $ M_{i}=\sup\{f(x):x_{i-1}\leq x \leq x_{i}\}$.
% 
% La \emph{suma inferior} y la \emph{suma superior} así definidas
% representan las sumas de las áreas de los rectángulos que tienen por
% bases los subintervalos de la partición, y por alturas los valores
% mínimo y máximo respectivamente de la función $f$sin los
% subintervalos considerados, tal y como se muestrasin la
% figura~\ref{g:sumassupinf}. Así, los valores de $L(P,f)$ y $U(P,f)$
% serán siempre menores y mayores respectivamente, que el área
% encerrada por la función $f$ y el eje de abscisassin el intervalo
% $[a,b]$.
% 
% \begin{figure}[htbp]
% \centering \subfigure[Suma inferior $L(P,f)$.]{
% \label{g:sumainferior}
% \scalebox{1}{\input{img/integrals/suma_inferior}}}\qquad\qquad
% \subfigure[Suma superior$U(P,f)$.]{
% \label{g:sumasuperior}
% \scalebox{1}{\input{img/integrals/suma_superior}}}
% \caption{Áreas medidas por las sumas superior e inferior
% correspondientes a una partición.} \label{g:sumassupinf}
% \end{figure}
% 
% Una función $f:[a,b]\rightarrow\mathbb{R}$ acotada es
% \emph{integrable}sin el intervalo $[a,b]$ si se verifica que:
% 
% \[
% \ \sup\{L(P,f): P \textrm{ partición de } [a,b]\}=\inf\{U(P,f): P
% \textrm{ partición de }[a,b]\}
% \]
% y ese número se designa por $\int_{a}^{b}f(x)\,dx$ o simplemente por
% $\int_{a}^{b}f$.
% 
% 
% \subsubsection*{Propiedades de la Integral}
% 
% \begin{enumerate}
% 
% \item \textbf{Linealidad}
% 
% Dadas dos funciones $f$ y $g$ integrablessin $[a,b]$ y $k \in
% \mathbb{R}$ se verifica que:
% 
% \[
% \
% \int_{a}^{b}(f(x)+g(x))\,dx=\int_{a}^{b}f(x)\,dx+\int_{a}^{b}g(x)\,dx
% \]
% y
% \[
% \ \int_{a}^{b}{kf(x)}\,dx=k\int_{a}^{b}{f(x)}\,dx
% \]
% 
% \item \textbf{Monotonía}
% 
% Dadas dos funciones $f$ y $g$ integrablessin $[a,b]$ y tales que
% $f(x)\leq g(x)$ $\forall x \in [a,b]$ se verifica que:
% 
% 
% \[
% \ \int_{a}^{b}{f(x)\,dx} \leq \int_{a}^{b}{g(x)\,dx}
% \]
% 
% \item \textbf{Acotación}
% 
% Si $f$ es una función integrablesin el intervalo $[a,b]$, existen
% $m,M\in\mathbb{R}$ tales que:
% 
% \[
% \ m(b-a)\leq\int_{a}^{b}{f(x)\,dx} \leq \ M(b-a)
% \]
% 
% \item \textbf{Aditividad}
% 
% Si $f$ es una función acotadasin $[a,b]$ y $c\in(a,b)$,sintonces $f$
% es integrablesin $[a,b]$ si y sólo si lo essin $[a,c]$ ysin $[c,b]$,
% verificándose además:
% 
% \[
% \ \int_{a}^{b}{f(x)\,dx} =
% \int_{a}^{c}{f(x)\,dx}+\int_{c}^{b}{f(x)\,dx}
% \]\\
% 
% \end{enumerate}
% 
% \subsubsection*{Teorema Fundamental del Cálculo}
% 
% Sea $f : [a,b]\rightarrow\mathbb{R}$ continua y sea $G$ una función
% continuasin $[a,b]$. Entonces $G$ es derivablesin $(a,b)$ y
% $G'(x)=f(x)$ para todo $x\in(a,b)$ si y sólo si:
% 
% \[
% \ G(x)-G(a) = \int_{a}^{x}f(t)\,dt
% \]
% 
% \subsubsection*{Regla de Barrow}
% 
% Si $f$ es una función continuasin $[a,b]$ y $G$ es continuasin
% $[a,b]$, derivablesin $(a,b)$ y tal que $G'(x)=f(x)$ para todo
% $x\in(a,b)$sintonces:
% 
% \[
% \  \int_{a}^{b}{f} = G(b)-G(a)
% \]
% 
% 
% De aquí se deduce que:
% 
% \[
% \  \int_{a}^{b}{f} = -\int_{b}^{a}{f}
% \]
% 
% 
% \subsection{Integrales impropias}
% 
% La integral $ \int_{a}^{b}{f(x)\,dx}$ se llama \emph{impropia} si el
% intervalo $(a,b)$ no está acotado o si la función $f(x)$ no está
% acotadasin el intervalo $(a,b)$.
% 
% Si el intervalo $(a,b)$ no está acotado, se denomina integral
% impropia de primera especie mientras que si la función no está
% acotadasin el intervalo se denomina improper integral de segunda
% especie.
% 
% \subsection{Cálculo de áreas}
% Una de las principales aplicaciones de la integral es el cálculo de
% áreas.
% 
% \subsubsection*{Área de una región planasincerrada por dos curvas}
% 
% Si $f$ y $g$ son dos funciones integrablessin el intervalo $[a,b]$ y
% se verifica que $g(x)\leq f(x)$ $\forall x\in[a,b]$,sintonces el
% área de la región plana limitada por las curvas $y=f(x)$, $y=g(x)$,
% y las rectas $x=a$ y $x=b$ viene dada por:
% 
% \[
% \ A = \int_{a}^{b}{(f(x)- g(x))\,dx}
% \]\\
% 
% \noindent \textbf{Observaciones}
% 
% \begin{enumerate}
% 
% \item El intervalo $(a,b)$ puede ser infinito y la definición sería análoga, perosin ese caso es preciso que la improper integral sea convergente.
% 
% \item Si $f(x)\geq0$ y $g(x)=0$ al calcular la integralsintre $a$ y $b$ se obtiene el áreasincerrada por la función $f(x)$ y el eje de abscisassintre las rectas verticales $x=a$ y $x=b$ (figura~\ref{g:integral_definida}).
% 
% \begin{figure}[h!]
% \begin{center}
% \scalebox{1}{\input{img/integrals/integral_definida}}
% \caption{Cálculo de áreasincerrada por la función $f(x)$ y el eje de
% abscisassintre las rectas verticales $x=a$ y $x=b$  mediante la
% definite integral.} \label{g:integral_definida}
% \end{center}
% \end{figure}
% 
% \item Si $f(x)\geq 0$ $\forall x\in[a,c]$ y $f(x)\leq 0$ $\forall x\in[c,b]$, el área de la región planasincerrada por $f$, las rectas verticales $x=a$ y $x=b$ y el eje de abscisas se calcula
% mediante:
% \[
% \ A= \int_{a}^{c}{f(x)\,dx} - \int_{c}^{b}{f(x)\,dx}.
% \]
% 
% \item Si las curvas $y=f(x)$ e $y=g(x)$ se cortansin los puntos de abscisas $a$ y $b$, no cortándosesin ningún otro punto cuya abscisa esté comprendidasintre $a$ y $b$, el áreasincerrada por dichas curvassintre esos puntos de corte puede calcularse
% mediante:
% \[
% \ A= \int_{a}^{b}{|f(x)-g(x)|dx}
% \]
% \end{enumerate}
% 
% 
% \subsection{Cálculo de Volúmenes}
% 
% \subsubsection*{Volumen de un sólido}
% Si se considera un cuerpo que al ser cortado por un plano
% perpendicular al eje $OX$ da lugar,sin cada punto de abscisa $x$, a
% una sección de área $A(x)$, el volumen de dicho cuerpo comprendido
% entre los planos perpendiculares al eje $OX$sin los puntos de
% abscisas $a$ y $b$ es:
% 
% \[
% \ V = \int_{a}^{b}{A(x)\,dx}
% \]
% 
% \subsubsection*{Volumen de un cuerpo de revolución}
% Si se hace girar la curva $y=f(x)$ alrededor del eje $OX$ se genera
% un sólido de revolución cuyas secciones perpendiculares al eje $OX$
% tienen áreas $A(x)=\pi(f(x))^{2}$, y cuyo volumen comprendidosintre
% las abscisas $a$ y $b$ será:
% 
% \[
% \ V = \int_{a}^{b}{\pi(f(x))^{2}\,dx}=
% \pi\int_{a}^{b}{(f(x))^{2}\,dx}
% \]
% 
% 
% En general, el volumen del cuerpo de revoluciónsingendrado al girar
% alrededor del eje $OX$ la región plana limitada por las curvas
% $y=f(x)$, $y=g(x)$ y las rectas verticales $x=a$ y $x=b$ es:
% 
% \[
% \ V = \int_{a}^{b}{\pi|(f(x))^{2}-(g(x))^{2}|\,dx}
% \]
% 
% De manera análoga se calcula el volumen del cuerpo de revolución
% engendrado por la rotación de una curva $x=f(y)$ alrededor del eje
% $OY$,sintre los planos $y=a$ e $y=b$, mediante:
% 
% \[
% \ V = \int_{a}^{b}{\pi(f(y))^{2}dy} = \pi \int_{a}^{b}{(f(y))^{2}dy}
% \]


\section{Solved exercises}

\begin{enumerate}[leftmargin=*]
\item Compute the following integrals:
\begin{enumerate}
\item $ \int{x^2 \log x\,dx}$
\begin{indication}
\begin{enumerate}
\item Enter the expression \command{x\^{}2 log(x)} in the Algebra window.
\item Select the menu \menu{Calculus > Integrate} or click the button \button{Find Integral}.
\item In the dialog shown check the option \option{Indefinite}, enter the constant \command{C} in the field \field{Constant} and click the button \button{Simplify}.
\end{enumerate}
A faster way is entering the expression \command{INT(x\^{}2 log(x), x, C)} in the Algebra window and clicking the button \button{Simplify}.
\end{indication}

\item $\displaystyle \int \frac{\log(\log x)}{x}\,dx$
\begin{indication}
Enter the expression \command{INT(log(log(x)), x, C)} in the Algebra window and click the button \button{Simplify}.
\end{indication}

\item $\displaystyle \int \frac{5x^{2}+4x+1}{x^{5}-2x^{4}+2x^{3}-2x^{2}+x}\,dx$
\begin{indication}
Enter the expression \command{INT((5x\^{}2+4x+1)/(x\^{}5-2x\^{}4+2x\^{}3-2x\^{}2+x), x, C)} in the Algebra window and click the button \button{Simplify}.
\end{indication}

\item $\displaystyle \int \frac{6x+5}{(x^{2}+x+1)^{2}}\,dx$
\begin{indication}
Enter the expression \command{INT((6x+5)/((x\^{}2+x+1)\^{}2), x, C)} in the Algebra window and click the button \button{Simplify}.
\end{indication}
\end{enumerate}


\item Compute the following definite integrals:
\begin{enumerate}
\item $\displaystyle \int_{-\frac{1}{2}}^0 \frac{x^{3}}{x^{2}+x+1}\,dx$
\begin{indication}
\begin{enumerate}
\item Enter the expression \command{x\^{}3/(x\^{}2+x+1)} in the Algebra window.
\item Select the menu \menu{Calculus > Integrate} or click the button \button{Find Integral}.
\item In the dialog shown check the option \option{Definite}, enter \command{-1/2} in the field \field{Lower Limit}, enter \command{0} in the field \field{Upper Limit} and click the button \button{Simplify}.
\end{enumerate}
A faster way is entering the expression \command{INT(x\^{}3/(x\^{}2+x+1), x, -1/2, 0)} in the Algebra window and clicking the button \button{Simplify}.
\end{indication}

\item $\displaystyle \int_2^4 \frac{\sqrt{16-x^{2}}}{x}\,dx$
\begin{indication}
Enter the expression \command{INT(sqrt(16-x\^{}2)/x, x, 2, 4)} in the Algebra window and click the button \button{Simplify}.
\end{indication}

\item $\displaystyle \int_0^{\frac{\pi}{2}} \frac{dx}{3+\cos(2x)}$
\begin{indication}
Enter the expression \command{INT(sqrt(1/(3+cos(2x), x, 0, pi/2)} in the Algebra window and click the button \button{Simplify}.
\end{indication}
\end{enumerate}


\item Compute the following improper integral $\int_2^{\infty} x^2e^{-x}\,dx$.
\begin{indication}
Enter the expression \command{INT(x\^{}2 exp(-x), x, 2, inf)} in the Algebra window and click the button \button{Simplify}.
\end{indication}


\item Plot the graph of the parabola $y=x^2-7x+6$ and compute the area between the parabola and the horizontal axis, limited by the lines $x=2$ and $x=7$.
\begin{indication}
To plot the graph of the parabola:
\begin{enumerate}
\item Define the function in the Algebra window entering the expression \command{f(x):=x\^{}2-7x+6}.
\item Open a new graphic window with the menu \menu{Window > New 2d-plot Window} and select the menu \menu{Window -> Tile Vertically} to see the Algebra and the graphic windows at the same time.
\item Click the button \button{Plot} in the graphic window.
\end{enumerate}
To compute the area enter the expression \command{INT(ABS(f(x)), x, 2, 7)} in the Algebra window and click the button \button{Simplify}.

To plot the area enter the expression \command{x>2 $\wedge$ x<7 $\wedge$ y>MIN(0,f(x)) $\wedge$ y<MAX(0,f(x))} in the Algebra window and click the button \button{Plot} in the graphic window.
\end{indication}


\item Compute and plot the area between the graphs of the functions $\sin x$ and $\cos x$ in the interval $[0,2\pi]$.
\begin{indication}
To plot the area:
\begin{enumerate}
\item Define the first function in the Algebra window entering the expression \command{f(x):=sin x}.
\item Open a new graphic window with the menu \menu{Window > New 2d-plot Window} and select the menu \menu{Window -> Tile Vertically} to see the Algebra and the graphic windows at the same time.
\item Click the button \button{Plot} in the graphic window.
\item Define the second function in the Algebra window entering the expression \command{f(x):=cos x} and click the button \button{Plot} in the graphic window.
\item Enter the expression \command{x>0 $\wedge$ x<2pi $\wedge$ y>MIN(f(x),g(x)) $\wedge$ y<MAX(f(x),g(x))} in the Algebra window and click the button \button{Plot} in the graphic window.
\end{enumerate}
To compute the area enter the expression \command{INT(ABS(f(x)-g(x)), x, 0, 2pi)} in the Algebra window and click the button \button{Simplify}.
\end{indication}



\item Plot the region of the first quadrant limited by the parabola $y^2=8x$ the horizontal axis and the line $x=2$. 
Compute the volume of the solid of revolution generated rotating the area around the horizontal axis. 
\begin{indication}
To plot the region:
\begin{enumerate}
\item Define the function in the Algebra window entering the expression \command{f(x):=sqrt(8x)}.
\item Open a new graphic window with the menu \menu{Window > New 2d-plot Window} and select the menu \menu{Window -> Tile Vertically} to see the Algebra and the graphic windows at the same time.
\item Click the button \button{Plot} in the graphic window.
\item Enter the expression \command{x>0 $\wedge$ x<2 $\wedge$ y>0 $\wedge$ y<f(x)} in the Algebra window and click the button \button{Plot} in the graphic window.
\end{enumerate}
To compute the volume of the solid of revolution enter the expression \command{INT(pi f(x)\^{}2, x, 0, 2)} in the Algebra window and click the button \button{Simplify}.
\end{indication}

\end{enumerate}


\section{Proposed exercises}
\begin{enumerate}[leftmargin=*]
\item Compute the following integrals:
\begin{enumerate}
\item $\displaystyle \int \frac{2x^3+2x^2+16}{x(x^2+4)^2\,dx}$
\item $\displaystyle \int \frac{1}{x^2\sqrt{4+x^2}}\,dx$
\end{enumerate}

\item Compute the area limited by the parabola $y=9-x^2$ and the line $y=-x$.

\item Compute the area limited by the graph of the function $y=e^{-|x|}$ and its asymptote.

\item Plot the region limited by the parabola $y=2x^2$, the lines $x=0$, $x=5$ and the horizontal axis. 
Compute the volume of the solid or revolution generated rotating that region around the horizontal axis.

\item Compute the volume of the solid of revolution generated rotating around the vertical axis the region limited by the parabola $y^2=8x$ and the line $x=2$.
\end{enumerate}
